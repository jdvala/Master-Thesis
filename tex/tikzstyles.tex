\usepackage{pgfplots}
\usepackage{tikz}
	\usetikzlibrary{arrows,positioning,backgrounds,fit,trees} 
	\usetikzlibrary{fadings,shapes.geometric}
	\usetikzlibrary{decorations,scopes,calc,decorations.pathreplacing}

\ifgerman{
	\pgfplotsset{tick label style={/pgf/number format/1000 sep=.,/pgf/number format/use comma}}
}

% Tortendiagramme
\newcommand{\slice}[4]{
  \pgfmathparse{0.5*#1+0.5*#2}
  \let\midangle\pgfmathresult

  % slice
  \draw[thick,
	%fill=background
	] (0,0) -- (#1:1) arc (#1:#2:1) -- cycle;

  % outer label
  \node[label=\midangle:#4] at (\midangle:1) {};

  % inner label
  \pgfmathparse{min((#2-#1-10)/110*(-0.3),0)}
  \let\temp\pgfmathresult
  \pgfmathparse{max(\temp,-0.5) + 0.8}
  \let\innerpos\pgfmathresult
  \node at (\midangle:\innerpos) {#3};
}
\newcommand{\mypiechart}[2]{
	\begin{tikzpicture}[scale=#1]
		\newcounter{a}
		\newcounter{b}
		\foreach \p/\t in {#2}
			{
				\setcounter{a}{\value{b}}
				\addtocounter{b}{\p}
				\slice{\thea/100*360}
							{\theb/100*360}
							{\p\%}{\t}
			}
	\end{tikzpicture}
}
