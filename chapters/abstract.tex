\ifgerman{\chapter*{Inhaltsangabe}}{\chapter*{Abstract}}
The plethora of legal corpora available online can be overwhelming and hard to comprehend for a non-domain expert. Big law firms also need to organize and update legal documents to keep up with changes in regulations and legislation. These law firms employ experts who help them navigate and find useful information for legal documents in a timely manner. \gls{AI} can be helpful in the organization and exploration of these documents \cite{merkl1997exploration}.  Classifying these legal texts into higher level categories using machine learning and deep learning can help in the organization and navigation of these huge corpora for big firms and the non-domain expert. The diversity, the advantages and the drawbacks of these algorithms make choosing the algorithm a time-consuming and challenging process.

This thesis goes into investigating a few popular machine learning and deep learning algorithms with different configurations on EUR-Lex Summaries legal data for text classification. It also examines the performance of general-purpose resources used by deep learning algorithms on legal domain-specific data and the performance benefits of using multilingual data which is widely available for the legal domain. These experiments help us in exploring the viability of these algorithms in domain-specific settings and improve our decision-making process.  For the investigation, three different research questions are formulated, first comparing a supervised machine learning algorithm  \gls{SVM} to a deep learning algorithm \gls{BiLSTM}, second comparing the general-purpose word embeddings to the domain-specific word embeddings for training \gls{BiLSTM} and third comparing performance evaluation of classifiers using multilingual data to monolingual ones in \gls{BiLSTM}.

Furthermore, with the limited labeled legal domain dataset, and class imbalance, several alternate training, and evaluation strategies were formulated. The training is done on sentences of the document with and without clustering, and the evaluation is done of sentences as well as on documents through the combination of predictions from the sentence. During the assessment, it is observed that adding more languages in case of a \gls{BiLSTM} indeed increases the performance of the classifier.































